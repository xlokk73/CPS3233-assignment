\documentclass[a4paper, 12pt]{article}
\usepackage{graphicx}
\usepackage{listings}

\setlength\parindent{24pt}

\lstset{language=python,breaklines=true, frame=single}

\begin{document}
\begin{figure}
    \centering
    \includegraphics[width=1\textwidth]{Logo}
\end{figure}

\title{CPS3233 Assignment Report}
\author{Manwel Bugeja}
\date{\today}
\maketitle
  
\tableofcontents
\newpage

\section{Elevator System Specification}
In this section, the specifications of the elevator system are expressed in several different formal notations. The notations are Finite State Automata (FSAs), Regular Expressions (RE), Timed Automata (TAs) and Duration Calculus (DC). From the ones listed, TAs and DC are the capable of expressing timed events. \\

\subsection{Finite State Automata}
\subsection{Regular Expressions}
\subsection{Timed Automata}
\subsection{Duration Calculus}


\section{Runtime Verification}

\section{Model-Based Testing}

\section{Runtime Verification and Testing}

\bibliographystyle{abbrv}
 \bibliography{references}

\end{document}
